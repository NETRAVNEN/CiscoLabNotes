% Declare Document Class
\documentclass[a4paper,12pt,twoside,twocolumn]{book}

\setlength{\columnsep}{2em}

% Latex Packages
\usepackage[T1]{fontenc}
\usepackage[utf8]{inputenc}
\usepackage{lmodern}
\usepackage{geometry}
\usepackage{listings}
\usepackage{color}
\usepackage{graphicx}
\usepackage{float}
\usepackage[english]{babel}
\usepackage{fancyhdr}
\usepackage{wrapfig}
\usepackage{array}
\usepackage{lipsum}
\usepackage{fancybox}
\usepackage{varwidth}

% Define color
\definecolor{codegreen}{rgb}{0,0.6,0}
\definecolor{codegray}{rgb}{0.5,0.5,0.5}
\definecolor{codepurple}{rgb}{0.58,0,0.82}
\definecolor{backcolour}{rgb}{0.95,0.95,0.92}

\lstdefinestyle{mystyle}{
    backgroundcolor=\color{backcolour},   
    commentstyle=\color{codegreen},
    keywordstyle=\color{magenta},
    numberstyle=\tiny\color{codegray},
    stringstyle=\color{codepurple},
    basicstyle=\footnotesize,
    breakatwhitespace=false,         
    breaklines=true,                 
    captionpos=b,                    
    keepspaces=true,                 
    numbers=left,                    
    numbersep=5pt,                  
    showspaces=false,                
    showstringspaces=false,
    showtabs=false,                  
    tabsize=4
}

\lstset{style=mystyle}

% Path where images are located relative
% to the file main.tex
\graphicspath{{img/}{figures/}}

% Custom commands
% Usage: \pic[<pct-of-columnwidth>]{<path-to-file>}
\newcommand{\pic}[2][50]{
\begin{center}
    \transparent{0.4}
    \includegraphics[width=0.#1\columnwidth]{#2}
\end{center}
}
% Usage: \fig{<path-to-file>}{<label>}{<caption>}
\newcommand{\fig}[3]{
\begin{figure}[h]
    \centering
    \includegraphics[width=0.95\columnwidth]{#1}
    \caption{#3}
    \label{fig:#2}
\end{figure}
}

\newcommand{\notice}[2]{%
    \shadowbox{%
        \begin{varwidth}{\linewidth}
            \texttt{\textbf{#1}}\\
            #2
        \end{varwidth}
    }
}

% In which order to look after images in
% declared graphicspath{}'s
% 1. Low-quality JPG
% 2. Med-quality PNG
% 3. High-quality PDF
\DeclareGraphicsExtensions{.jpg,.png,.pdf}

% Package Params
\geometry{a4paper,margin=4em}

%\setlength{\parindent}{4em}
%\setlength{\parskip}{1em}
%\renewcommand{\baselinestretch}{2.0}

% Define fancy header and footer
\pagestyle{fancy}
\fancyhf{}
\fancyhead[LE,RO]{ZBC}
\fancyhead[RE,LO]{\rightmark}
\fancyfoot[CE,CO]{\leftmark}
\fancyfoot[LE,RO]{\thepage}

% width of header and footer rule is by default 0px.
\renewcommand{\headrulewidth}{2pt}
\renewcommand{\footrulewidth}{1pt}

% Use the roman numeric system for pagenumbers
\pagenumbering{roman}

%%%%%%%%%%%%%%%%%%%%%%%%%%%%%%%%%%%%%%%%%%%%%%%%%%%%%%
%                                                    %
% BEGIN DOCUMENT                                     %
%                                                    %
%%%%%%%%%%%%%%%%%%%%%%%%%%%%%%%%%%%%%%%%%%%%%%%%%%%%%%

\begin{document}

% Which info to insert on the title page
\title{r17dinh409}
\author{Christoffer Hansen <zbcchhan11 at zbc.dk>}
\date{May 22 - June 30, 2017}
% Make title page contents
\maketitle

\tableofcontents

% Define length between paragrahps
\setlength{\parskip}{1em}
% Define lineheight
\renewcommand{\baselinestretch}{1.15}

%%%%%%%%%%%%%%%%%%%%%%%%%%%%%%%%%%%%%%%%%%%%%%%%%%%%%%
%                                                    %
% BEGIN CHAPTER: Base Configuration                  %
%                                                    %
%%%%%%%%%%%%%%%%%%%%%%%%%%%%%%%%%%%%%%%%%%%%%%%%%%%%%%

\chapter{Base Configuration}

\section{Cisco Lab}

% <!-- ROUTER -->

\subsection{Router}
\subsubsection{File: base.cfg}
%\lstinputlisting[language=tcl]{code/router/base.cfg}
\subsubsection{File: reset.tcl}
%\lstinputlisting[language=tcl]{code/router/reset.tcl}

\newpage

% <!-- LAYER 3 SWITCH -->

\subsection{Layer 3 Switch}
\subsubsection{FILE: base.cfg}
\lstinputlisting[language=tcl]{code/l3switch/base.cfg}
\subsubsection{FILE: reset.tcl}
\lstinputlisting[language=tcl]{code/l3switch/reset-tcl.txt}
\subsubsection{FILE: resetvlans.tcl}
\lstinputlisting[language=tcl]{code/l3switch/resetvlans-tcl.txt}

\newpage

% <!-- LAYER 2 SWITCH -->

\subsection{Layer 2 Switch}
\subsubsection{FILE: base.cfg}
\lstinputlisting[language=tcl]{code/l2switch/base.cfg}
\subsubsection{FILE: reset.tcl}
\lstinputlisting[language=tcl]{code/l2switch/reset-tcl.txt}
\subsubsection{FILE: resetvlans.tcl}
\lstinputlisting[language=tcl]{code/l2switch/resetvlans-tcl.txt}

%%%%%%%%%%%%%%%%%%%%%%%%%%%%%%%%%%%%%%%%%%%%%%%%%%%%%%
%                                                    %
% BEGIN CHAPTER: Protocols                           %
%                                                    %
%%%%%%%%%%%%%%%%%%%%%%%%%%%%%%%%%%%%%%%%%%%%%%%%%%%%%%

\chapter{Protocols}

\section{Routing}

\subsection{OSPF}
\subsection{IS-IS}
\subsection{EIGRP}
\subsection{RIP}
\subsection{Static}
\subsection{BGP}

\newpage

\section{VLAN}

\subsection{VTP}
\fig{vtp/implementing-vtp}{imp-vtp1}{VTP}

\subsubsection{VTP Modes}
The tree modes a VTP \textit{enabled} device can operate are
\begin{itemize}
    \item Transparent
    \item Server
    \item Client
\end{itemize}
Of course you can \textit{disable} VTP altogether.

Key things to be aware of \textit{before} enabling VTP in your environment is to make double sure of only having 1 VTP domain. \textbf{If} 2 or more VTP domains exists. Be triple sure to separate them! As to avoid having an VTP server DB overridden with data from another VTP domain.

The three VTP modes \textit{operates} as follow
\begin{itemize}
    \item Transparent
    \begin{itemize}
        \item Creates, modifies and deletes \textit{local} vlans only
        \item Forwards advertisements
        \item Does \textit{not} synchronizes vlan configurations.
    \end{itemize}
    \item Server
    \begin{itemize}
        \item Creates, modifies and deletes vlans
        \item Sends and forwards advertisements
        \item Synchronizes vlan configurations
    \end{itemize}
    \begin{itemize}
        \item Cannot create, modify or delete vlans
        \item Send and forwards advertisements
        \item Synchronizes vlan configurations
    \end{itemize}
\end{itemize}

\subsubsection{VTP Announcement}
VTP operates with announcements sent out in intervals. Summarized it amounts to
\begin{itemize}
    \item 1 \textit{summary} announcement per 5th minute from the server
    \item The summary announcement informs clients of the current revision
    \item An announcement is sent out \textit{on the spot} when a change has been made on the VTP server
\end{itemize}

Do remember it is \textbf{only} the VTP server which has the vlan configuration stored \textbf{on disk}. All device clients and transparent nodes do only store the vlans delegated by VTP in memory.

\subsubsection{Common Issues}
\begin{itemize}
    \item Different/Incompatible VTP versions
    \item Wrong password
    \item Incorrect mode name
    \item No server set (all devices configured in transparent/client/vtp disabled mode)
\end{itemize}

\subsubsection{VTP Versions}
\begin{itemize}
    \item Version 1
    \item Version 2
    \begin{itemize}
        \item Version-dependent	transparent	mode
        \item Consistencycheck
        \item Token ring support
        \item Unrecognized type-length-value support
    \end{itemize}
    \item Version 3 (not "yet" common)
    \begin{itemize}
        \item Extended VLAN support: Allow ranges are 1-1005,1018-2095. Not mentioned vlans ranges up to 4095 is still reserved.
        \item Domain name is not automatically learned.
        \item Better security.
        \item Better database propagation.
        \item MST now supported.
    \end{itemize}
\end{itemize}

\subsubsection{VTP Pruning}
The art of only allowing the vlan traffic to flow on \textit{necessary} links.

This means if there are no clients in a vlan on a device. Then no traffic for the inactive vlans is send down-/upstream on the link in question.
\fig{vtp/vtp-pruning}{vtpruning1}{VTP Pruning}

\subsubsection{Security}
It is \textbf{strongly} recommended to enable the security features supported in VTP.

\textbf{Password:} MD5 hashing, Case-sensitive, Length between 8 and 64 chars.

\notice{VTP Scaling}{
As the network grows and grows and grows and grows some more over long/short timespans.
You will \textbf{for certain} come to cross-rode, where you \textbf{must} consider to
go away from using VTP in the network. The problems of managing an elderly network and
wiping and re-introducing nodes in the network. You \textbf{will} face the issue of a
wiped vlan database from the VTP domain.
}

\subsubsection{Example configuration}
\lstinputlisting{code/vtp/example.cfg}
    
\newpage

\section{Spanning Tree}

\subsection{STP}
\subsection{PVST}
\subsection{RPVST+}
\subsection{MTP}


%%%%%%%%%%%%%%%%%%%%%%%%%%%%%%%%%%%%%%%%%%%%%%%%%%%%%%
%                                                    %
% BEGIN CHAPTER: Internet                            %
%                                                    %
%%%%%%%%%%%%%%%%%%%%%%%%%%%%%%%%%%%%%%%%%%%%%%%%%%%%%%

\chapter{Internet}

\section{BGP}

%%%%%%%%%%%%%%%%%%%%%%%%%%%%%%%%%%%%%%%%%%%%%%%%%%%%%%
%                                                    %
% BEGIN LIST OF FIGURES                              %
%                                                    %
%%%%%%%%%%%%%%%%%%%%%%%%%%%%%%%%%%%%%%%%%%%%%%%%%%%%%%

\renewcommand{\listfigurename}{List of plots}
\listoffigures

%%%%%%%%%%%%%%%%%%%%%%%%%%%%%%%%%%%%%%%%%%%%%%%%%%%%%%
%                                                    %
% BEGIN LIST OF TABLES                               %
%                                                    %
%%%%%%%%%%%%%%%%%%%%%%%%%%%%%%%%%%%%%%%%%%%%%%%%%%%%%%

\renewcommand{\listtablename}{Tables}
\listoftables

%%%%%%%%%%%%%%%%%%%%%%%%%%%%%%%%%%%%%%%%%%%%%%%%%%%%%%
%                                                    %
% BEGIN REFERENCES                                   %
%                                                    %
%%%%%%%%%%%%%%%%%%%%%%%%%%%%%%%%%%%%%%%%%%%%%%%%%%%%%%

\bibliographystyle{unsrt}
\bibliography{unsrt}

%%%%%%%%%%%%%%%%%%%%%%%%%%%%%%%%%%%%%%%%%%%%%%%%%%%%%%
%                                                    %
% END DOCUMENT                                       %
%                                                    %
%%%%%%%%%%%%%%%%%%%%%%%%%%%%%%%%%%%%%%%%%%%%%%%%%%%%%%

\end{document}
