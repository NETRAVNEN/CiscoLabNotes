%%%%%%%%%%%%%%%%%%%%%%%%%%%%%%%%%%%%%%%%%%%%%%%%%%%%%%
%                                                    %
% SECTION BEGIN spanning tree protocol               %
%                                                    %
%%%%%%%%%%%%%%%%%%%%%%%%%%%%%%%%%%%%%%%%%%%%%%%%%%%%%%

\section{Spanning Tree}

Spanning Tree exists for the \textbf{sole} reason to save "your" network and all the broadcast storms an network engineer having a bad day can by mistake create!

STP comes from the above desire where redundancy was wanted but no protocol existed before \gls{stp} to help in this regard.

\begin{table}[h]
    \centering
    \caption{Spanning Tree standards}
    \label{stpstandards}
    \resizebox{\columnwidth}{!}{%
        \begin{tabular}{|l|l|l|l|l|}
            \hline
            \textbf{} & \textbf{Standard} & \textbf{Ressource Usage} & \multicolumn{2}{l|}{\textbf{Convergence}} \\ \hline
            \gls{cst} & 802.1D & Low & Slow & All vlans \\ \hline
            \gls{pvst+} & Cisco & High & Slow & Per vlan \\ \hline
            \gls{rstp} & 802.1w & So-so (Med.) & Fast & All vlans \\ \hline
            \gls{rpvst+} & Cisco & On-the-double (V.High) & Fast & Per vlan \\ \hline
            \gls{mst} & 802.1s & Med. - High & Fast & Vlan list \\ \hline
        \end{tabular}%
    }
\end{table}

\subsection{Port Roles}

When a switch is enabled for Spanning Tree. One of the following roles will have been assumed by any port on the switch in question.

\begin{itemize}
    \item \textbf{Root port:} Only 1 port on any switch (non-counting the root bridge!). Is always the port with the lowest metric (aka. best path) to the root bridge.
    \begin{itemize}
        \item The upstream/-link port closest to the root bridge on all switches apart from the root bridge.
    \end{itemize}
    \item \textbf{Designated port:} A designated port is the port on any segment closest to the root bridge and forwarding traffic.
    \begin{itemize}
        \item The port on any switch in downstream direction closet to the root bridge.
    \end{itemize}
    \item \textbf{\textit{Non}-designated port:} Put in blocking mode and not currently forwarding traffic.
    \begin{itemize}
        \item All switch ports which did not get elected as the root or designated port.
    \end{itemize}
    \item \textbf{Disabled port:} The port has been one-way-or-another shut down.
\end{itemize}

\subsubsection{specific port roles}
\begin{itemize}
    \item \textbf{Alternative port} is an active port in network with an alternative path to the root bridge. A port in alternative mode will remain active but \textit{discards} all traffic until the the current designated path fails.
    \item \textbf{Backup port} is running in active mode and \textit{discards} all traffic it recieves until the current designated port on the segment the backup port is connected to, fails.
\end{itemize}

Election of ports goes in order of the following values (low is best): 1) root bridge id, 2) lowest path cost to root bridge, 3) sender bridge id, 4) sender port bridge id

%%%%%%%%%%%%%%%%%%%%%%%%%%%%%%%%%%%%%%%%%%%%%%%%%%%%%%
%                                                    %
% SECTION BEGIN standards                            %
%                                                    %
%%%%%%%%%%%%%%%%%%%%%%%%%%%%%%%%%%%%%%%%%%%%%%%%%%%%%%

\subsection{Standards}

\begin{itemize}
    \item \gls{stp} {\scriptsize Spanning Tree Protocol}
    \begin{itemize}
        \item \gls{ieee} 802.1D
        \item Was created in a time where bridged networks was the norm.
        \item Supports a single vlan/lan.
    \end{itemize}
    \item \gls{cst} {\scriptsize Common Spanning Tree}
    \begin{itemize}
        \item An evolution of \gls{stp}
        \item \gls{cst} still only supports one \gls{stp} instance.
        \item But \gls{cst} do thou in contrast to \gls{stp} support \textit{multiple} vlans.
    \end{itemize}
    \item \gls{pvst} {\scriptsize Per Vlan Spanning Tree}
    \begin{itemize}
        \item Now obsolute and succeded by \gls{pvst+}
    \end{itemize}
    \item \gls{pvst+} {\scriptsize Per Vlan Spanning Tree Plus}
    \begin{itemize}
        \item Runs an instance of \gls{stp} per vlan.
        \item Can guarante better utilization of available network bandwidth.
        \item Root bridge and port priorities can be configured per vlan.
        \item Uses the term alternate for nondesignated port.
    \end{itemize}
    \item \gls{rstp} {\scriptsize Rapid Spanning Tree Protocol}
    \begin{itemize}
        \item \gls{ieee} 802.1w
        \item A future development of the original 802.1D standard meant to provide faster convergance. As the original \gls{stp} standard wasn't actually that fast.
    \end{itemize}
    \item \gls{rpvst+} {\scriptsize Rapid Per Vlan Spanning Tree Plus}
    \begin{itemize}
        \item A cisco implementation of \gls{rstp} based upon \gls{pvst+}.
    \end{itemize}
    \item \gls{mst} {\scriptsize Multiple Spanning Tree}
    \begin{itemize}
        \item Originally a cisco developed protocol. \gls{mst} has since been developed as an \gls{ieee} standard.
        \item \gls{mst} can as \gls{cst} map multiple vlans to a single \gls{stp} instance.
        \item \gls{mst} \textit{differently} than \gls{cst} supports multiple \gls{stp} instances.
        \item Fx. Instance 1: Vlan 1-99, Instane 2: Vlan 100-199.
    \end{itemize}
\end{itemize}

\subsection{Features}

\subsubsection{BPDU}
\textbf{B}ridge \textbf{P}rotocol \textbf{D}ata \textbf{U}nits is on cisco equipment sent out every 2 seconds and generally catogorizes into 2 categories:
\begin{itemize}
    \item \textit{Configuration} \gls{bpdu} used for \gls{stp} calculations and
    \item \textit{Topology change notifications} \gls{bpdu}s used to notify other network nodes of a change in the network.
\end{itemize}

Any network node with switchports and \gls{stp} + \gls{bpdu} enabled sends out \gls{bpdu} packets with the ports \gls{mac} as the src address. The destination \gls{mac} is is designated \gls{stp} multicast addr 01:80:C2:00:00:00.

\subsubsection{Root Bridge}
Using a \textbf{R}oot \textbf{B}rigde as the reference point for the \gls{stp} instance and calculation of root/designated/non-designated ports.\\This election process uses a pre-configured bridge priority (ranges from $0$ to $2^{16}$) (defaults to $2^{15}$). If a tie in priority is found the switch in possession of the lowest \gls{mac} address wins the root bridge election.

\begin{txt}
|<-------- bridge id 8 bytes --->|

+----------+---------------------+
| Bridge   |        MAC          |
| priority |       Address       |
+----------+---------------------+

|< 2 byte >|<----- 6 bytes ----->|
\end{txt}

\subsubsection{Port Cost}

\begin{table}[h]
    \centering
    \caption{Default port cost in spanning tree}
    \label{stpportcost}{!}{%
        \begin{tabular}{|l|l|}
            \hline
            \textbf{Link} & \textbf{Default Cost} \\ \hline
            10 Gbps & 1 \\ \hline
            1 Gbps & 4 \\ \hline
            100 Mbps & 19 \\ \hline
            10 Mbps & 100 \\ \hline
        \end{tabular}%
    }
\end{table}

\fig{spanningtree/portroles}{stpportroles}{Port Election}

\textit{\textbf{NB:} beware that when working with bundled links (aka. ether-/port-channel). Then the link cost will be calculated based upon the summarized bandwidth accross all links.}

\fig{spanningtree/portstates}{stpportstates}{Port States}

\section{Rapid Spanning Tree Protocol}

\fig{rstp/portroles}{rstpportroles}{Port Roles}

\fig{rstp/portlinktypes}{rstpportlinktypes}{Port link types}

Things to be aware of regarding \gls{rstp} port roles
\begin{itemize}
    \item \textbf{Shared} port state will only ever be present on segments where a hub is present.
    \item \textbf{Point-2-Point} port is connected to a single switch on the other end.
    \item \textbf{Edge} port roles is only ever connected to end devices. Status as Edge port is lost if a \gls{bpdu} is ever recieved.
\end{itemize}

%%%%%%%%%%%%%%%%%%%%%%%%%%%%%%%%%%%%%%%%%%%%%%%%%%%%%%
%                                                    %
% SECTION BEGIN port roles                           %
%                                                    %
%%%%%%%%%%%%%%%%%%%%%%%%%%%%%%%%%%%%%%%%%%%%%%%%%%%%%%

\section{Port roles}

\subsection{Fast port roles}
Cisco did on their part early on enhance the original spanning tree standard with some proprietary portroles that can (on cisco switch equipment) skip steps in the port role election process. And configure a \gls{stp} switchport to a specific behavior as described below:

\begin{itemize}
    \item PortFast
    \begin{itemize}
        \item Configures access port to transition directly to forwarding state.
        \item Improve convergence times of non-\gls{rstp}.
        \item Port does no forwan \gls{tcn} \gls{bpdu}s either.
        \item PortFast can be enabled either A) per port \textit{or} B) globally for all ports in access mode.
        \begin{enumerate}
            \item Per port: {\footnotesize Accesss port}\\\cliline{cisco-switch(config-if)# spanning-tree portfast}
            \item Per port: {\footnotesize Trunk port}\\\cliline{cisco-switch(config-if)# spanning-tree portfast trunk}
            \item Globally:\\\cliline{cisco-switch(config)# spanning-tree portfast default}
        \end{enumerate}
    \end{itemize}
    \item UplinkFast
    \begin{itemize}
        \item Enables fast uplink failover on access switch.
        \item Improve convergence times of non-\gls{rstp}.
        \item Enabled only with non-\gls{rstp}
        \item Integrated into Cisco's \gls{rstp} implementaion and enabled by defaut.
        \item Cisco proprietary
        \item Only works if switch has blocked ports
        \item Designed with switches in access layer as deployment target.
        \item Enabled for the entire switch. Cannot be enabled pr. vlan.
        \item \cliline{cisco-switch(config)# spanning-tree uplinkfast} enables the feature.
    \end{itemize}
    \item BackboneFast
    \begin{itemize}
        \item Enables fast convergence in distribution or core layer when \gls{stp} change occurs.
        \item Improve convergence times of non-\gls{rstp}.
        \item Enabled only with non-\gls{rstp}
        \item Integrated into Cisco's \gls{rstp} implementaion and enabled by default.
        \item Disabled by default
        \item \cliline{cisco-switch(config)# spanning-tree backbonefast} enables the feature.
        \item \textit{Scenario:} If switch needs searching new path root bridge. BackboneFast shortens process.
        \begin{enumerate}
            \item Switch will search for alternative path to root.
            \item If \gls{bpdu} recieved on blocked port. Port considered alternative path path to root.
            \item If alternate path identified. RQL{\footnotesize \textbf{R}equest \textbf{L}ink \textbf{B}locking} packets are out for identify either A) an alternative path to the root bridge \textit{or} B) an up-/downstream switch with a path to the root bridge.
        \end{enumerate}
    \end{itemize}
\end{itemize}

\subsection{Loop Prevention}

\begin{itemize}
    \item \gls{bpdu} Guard
    \begin{itemize}
        \item Disables the PortFast-enabled port if a \gls{bpdu} is received. The port goes into mode \texttt{err-disable}.
        \item Enable per port:\\\cliline{cisco-switch(config-if)# spanning-tree bpduguard enable}
        \item Enable globally for portfast enabled ports:\\\cliline{cisco-switch(config)# spanning-tree portfast bpduguard default}
    \end{itemize}
    \item \gls{bpdu} Filter
    \begin{itemize}
        \item Suppresses \gls{bpdu}s on ports
        \item Behaves differently depending if enabled
        \item A) globally \textit{or}
        \begin{enumerate}
            \item Affects all active portfast enabled ports, which \underline{don't} have a \gls{bpdu} port configuration.
            \item If \gls{bpdu} recieved on port, portfast and \gls{bpdu} filter is disabled.
            \item Sends \textbf{10} \gls{bpdu}s on startup. If \gls{bpdu} recieved in this timeframe \textit{same consequence as above} happens to the port.
            \item \cliline{cisco-switch(config-if)# spanning-tree bpdufilter enable}
        \end{enumerate}
        \item B) per-port:
        \begin{enumerate}
            \item Port ignores all recieved \gls{bpdu}s.
            \item Port sends no \gls{bpdu}s.
            \item \cliline{cisco-switch(config-if)# spanning-tree bpdufilter enable}
        \end{enumerate}
        \item Beware to \underline{only} enable \gls{bpdu} filter on ports connected to end hosts. Consequence if not followed \underline{can} result in creating bridging loops.
        \item Beware to \underline{only enable either} \gls{bpdu} guard \textbf{\textit{or}} filter. \footnote{Cisco recommendation}
    \end{itemize}
    \item Root Guard
    \begin{itemize}
        \item \st{Prevents external switches from becoming roots}
        \item If enabled, prevents any ports from becoming a root-port. Ports will remain as designated ports \textit{effectivily} preventing the switch becoming the root bridge.
        \item This, too, behaves in s similiar manner as \gls{bpdu} guard, putting the port in \texttt{err-disable} mode when a \gls{bpdu} packet is recieved on the port.
        \item Enabled per-port with\\\cliline{cisco-switch(config-if)# spanning-tree guard root}
    \end{itemize}
    \item Loop Guard
    \begin{itemize}
        \item Prevents an alternate port from becoming the designated port if no \gls{bpdu}s are received
        \begin{enumerate}
            \item Normally when cisco swicthes stop recieving \gls{bpdu}s ingress in a port. The port will go to listeting, learning, forwarding state equaling a loop.
            \item With Loop guard enabled the will go to \texttt{loop-inconsistent} blocking state instead.
        \end{enumerate}
        \item Enabled per-port\\\cliline{cisco-switch(config-if)# spanning-tree guard loop}
        \item Enabled globally\\\cliline{cisco-switch(config)# spanning-tree loopguard default} {\small only on p2p links}
        \item Works on per-vlan basis when \gls{pvst} is used.
        \item On ether-channel links with uni-directional link failures, loop guard will put put the whole ether-channel into loop-inconsistent state.
    \end{itemize}
    \item \textbf{Beware} root and loop guard is mutually exclusive
    \begin{itemize}
        \item Root guard works on designated ports and does not allow the ports to become \textit{non}-designated ports, where
        \item Loop guard works on \textit{non}-designated ports and does not allow the ports to become designated ports {\footnotesize though expiration of times}.
    \end{itemize}
\end{itemize}

\subsection{Link}

\begin{itemize}
    \item \gls{udld}
    \begin{itemize}
        \item Cisco proprietary feature.
        \item By default only enables on fiber optic links.
        \item Works by sending packes every 15 seconds (default timer). If not packet is recieved back, the port can either log (default) a messaage or actively try to re-establish the link (aggresive). 1 packet/second for 8 sec. is send. If non is returned the port will go to \texttt{err-disable} state.
        \item \cliline{cisco-switch(config)\# udld \{enable | aggresive\}}
        \item On ether-channel links with uni-directional link failures, udld will disable individual failed links.
        \item For the best protection. Aggresive mode is recommended.
        \item It is recommended to turn on udld in global conf mode.
    \end{itemize}
    \item FlexLinks
    \begin{itemize}
        \item Cisco proprietary feature.
        \item An alternate solution to running \gls{stp} in the environment.
        \begin{itemize}
            \item \gls{stp} is auto-disabled on interfaces running FlexLinks.
            \item Configured with 2 physical links with and active/backup configuration.
            \item Enables convergence time of less than 50 milliseconds.
        \end{itemize}
        \item FlexLinks is good alternative to running \gls{stp} in an environment with customers who you do \textit{not} want to run \gls{stp} with. Fx. Service Provider/Enterprise/Datacenter environment.
        \item Preemtion for FlexLinks is \textit{not} enabled-by-default.
        \begin{enumerate}
            \item Detects link failure.
            \item Moves any dynamic unicast \gls{mac} addresses learned on primary link to standby link.
            \item Moves standby link to forwarding state.
            \item Transmits dummy multicast packets over new active interface. {\small Dummy multicast packet format is as follows: \textbf{destination:} 01:00:0c:cd:cd:cd, \textbf{source:} \gls{mac} address of the hosts or ports on the newly active FlexLinks port}
        \end{enumerate}
        \item {\small \textbf{Note:}} Configuring FlexLinks outside of access layer switches can be very complex!
        \item Enabled FlexLinks on an interface: \\
            \cliline{cisco-switch(config)# interface fa0/1} \\
            \cliline{cisco-switch(config-if)# switchport backup interface fa0/2}
        \item \textbf{What} FlexLinks can be:
        \begin{enumerate}
            \item A physical port
            \item A Bundled link {\footnotesize (aka. ether-channel)}
            \item 1 FlexLink per physical/logical port
            \item Link speeds need not be the same
        \end{enumerate}
    \end{itemize}
\end{itemize}

\begin{table}[h]
    \centering
    \caption{UDLD|Loopguard compared}
    \label{udldloopguard}
    \resizebox{\columnwidth}{!}{%
        \begin{tabular}{|l|l|l|}
            \hline
            \thead{Functionality} & \thead{Loop guard} & \thead{UDLD} \\ \hline
            Action granularity & Per vlan & Per port \\ \hline
            \makecell{Protection against \gls{stp}\\failures caused by uni-directional\\ links} & \makecell{Yes, when enabled on all\\potential non-designated ports\\in redundant topology} & \makecell{Yes, when enabled on all\\links in redundant topology} \\ \hline
            \makecell{Protection against \gls{stp}\\failures caused by problem in\\software resulting in designated\\switch not sending \gls{bpdu}s} & Yes & No \\ \hline
            Protection against mis-wiring & No & Yes \\ \hline
        \end{tabular}%
    }
\end{table}

\fig{spanningtree/stpbestpractice}{stpbestpractice}{STP best practice}

\section{Multiple Spanning Tree}

\begin{itemize}
    \item \itemtitle{Known limitations}{Regarding the cisco world of things}
    \begin{enumerate}
        \item A maximum of 16 instances is supported. {\footnotesize From 0 to 15.}
    \end{enumerate}
    \item \textbf{Beware} that instance 0 is the \textit{I}nternal \textit{S}panning \textit{T}ree. And therefore cannot be configured for user-mapped Vlans.
    \item Aggregates the configured vlans into groups/instances/processes. This in turn provides lower resource utilization on switches. \dWinkey
    \item Backwards compatible with 802.1D \gls{stp}/802.1w/\gls{rstp} and Cisco \gls{pvst+}.
    \item Converges faster than \gls{pvrst+}.
    \item \itemtitle{Challenges}{Arises because of older hardware and the architecture of the protocol}
    \begin{enumerate}
        \item Operability with older/legacy hardware/equipment is not always possible.
        \item \textit{Of course} it is more complex compared to standard \gls{stp} (older) protocols. {\footnotesize Staff may require teachings of the way of the protocol.}
    \end{enumerate}
\end{itemize}

\begin{table}[h]
    \centering
    \caption{MST Attributes}
    \label{mstattr}
    \resizebox{\columnwidth}{!}{%
        \begin{tabular}{|l|l|}
            \hline
            \thead{Data} & \thead{What ?} \\ \hline
            32 bytes & alphanumeric configuration name \\ \hline
            2 bytes & configuration revision number \\ \hline
            Table of 4096 elements & \makecell{associates each of the potential\\4096 VLANs with an instance} \\ \hline
        \end{tabular}%
    }
\end{table}

\subsection{MST Regions}

It is the network admins job to propagate an even configuration to all switches in a single region by using \gls{cli} or \gls{snmp}. Currently IOS does not support any other options to do the job.

\begin{itemize}
    \item \itemtitle{Boundaries}{MST differs between regions by}
    \begin{enumerate}
        \item sending a digest computer from the Vlan-to-instance mapping table of the switch sending the digest.
        \item the characteristics of the \gls{mst} protocol for that single switch.
    \end{enumerate}
    \item if computed digest and \gls{mst} characteristics between switches is \textit{found matching}, the switches considers themselves part of the same \gls{mst} region.
    \item \textbf{Beware} that unlike \gls{vtp}, \gls{mst} does not automatically increase the configuration revision number. This \textit{has to be done} manually.
\end{itemize}

\fig{spanningtree/mstregions}{mstregions}{MST Regions all Vlans running mappen to the default instance 0.}

\fig{spanningtree/mstregions2}{mstregions2}{MST Regions vlans mapped to different instances.}

\pagebreak % avoids column break for the following text

\begin{txt}
|<-------- bridge id ----------->|

+----------+-----------+---------+
| Bridge   | Extended  | MAC     |
| priority | system ID | Address |
+----------+-----------+---------+

                /\
                ||
    \gls{mst} inst. no. carried in
      Ext. Syst. ID Area
\end{txt}

\subsection{Common Troubleshooting}

\begin{itemize}
    \item Duplex mismatch: Mismatch half/full duplex.
    \item Unidirectional link failure
    \item Frame corruption: Physical interfaces can be defective/with errors.
    \item Resource errors: High compute usage in the Control Plane. {\scriptsize Remember the Spanning Tree Process in fx. Cisco Catalyst switches is handled in software.}
    \item PortFast configuration error: Can lead to bridging loops when mis-wiring happens together with logical port configuration errors.
\end{itemize}
