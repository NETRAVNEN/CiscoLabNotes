\chapter{Fiber Optical stuff}

\texttt{This chapter is purely dedicated to things/stuff related to physical optical networks.}

\section{10 advantages of using optical fiber in networks}

\begin{quotation}
    Generally, sooner or later any network installer and planner reaches the limits of current network media – be it coaxial cables using HFC technology or EOC, or copper networks using ADSL or VDSL, or even Ethernet over Power networks. And yes, radio waves also have it’s limits as we will see further in this article. In general, here are 10 main advantages to chose optical fiber as transmission media for your networks:\cite{10advant21:online}
\end{quotation}

\fig{fiber/fiber-optic-cable-internal-structure}{fiber-internal-structure}{Fiber-optic cable - Internal Structure}

\begin{itemize}
    \item Long transmission distance
    \item Dielectric protection and construction
    \item Small size
    \item Light weight
    \item Relatively easy installation (but skills are needed though)
    \item Secure transmission (as oposed to wireless especially)
    \item EMI and RFI immunity (you have no distortions for example in CCTV signal)
    \item Low cost
    \item Unlimited Bandwidth
    \item Not attractive to fraud
\end{itemize}

\subsection{Discussion}

Fiber in general is a better idea when running cables (probably not to end users inside a building!). If single-mode is expensive and cable runs above 550 metres is not a requirement. Go with multi-mode (cheaper than single-mode).

You can get quite cheap \gls{mmf} optics if looking only at 1.25 \gls{gbps} speeds. If upgrading to 10 \gls{gbps} or even 40 \gls{gbps} speeds in your \gls{lan}. The it gets more expensive even using \gls{mmf} optics. But is is still cheaper than using \gls{smf} optics and fx. changing all internal optical cable wiring to single-mode from multi-mode.

So \gls{mmf} will get one a long way. (Unless \#CompanyPersonInCharge) decides on \gls{smf} is the way to go. Even thou the price-tag is higher.
