\chapter[QoS]{Quality-of-Service}

\gls{qos} is used to guarantie a minimum of service level to select applications. Often this encompasses \gls{voip} applications and \gls{av} applications being allocated the highest priority. It\tsq{s} not uncommon to allocate \gls{nmt} to the high priority queue, too.

Different mechanisms of handling access to network ressorces is used.
\begin{itemize}
    \item \itemhead{\gls{nac}}
    \begin{enumerate}
        \item Which applications has access to what level of network ressources.
    \end{enumerate}
    \item \itemhead{Traffic Control}
    \begin{enumerate}
        \item Scheduling of traffic,
        \item classifiying traffic,
        \item marking packets based upon priority,
        \item marking packets based upon shaping traffic.
    \end{enumerate}
\end{itemize}

\section{Concepts}

\begin{enumerate}
    \item Bandwidth,
    \item latency,
    \item jitter,
    \item realiability.
\end{enumerate}

\section[Congestion Mgmt]{Congestion Management}

There are different ways to do congestion management. Which is in it\tsq{s} essence sorting of packets when a link reaches full capacity usage in the outgoing direction.

\begin{enumerate}
    \item \gls{fifo}
    \item \gls{pq}
    \item \gls{cq}
    \item \gls{wfq}
    \begin{enumerate}
        \item Flow-based \gls{wfq},
        \item class-based \gls{wfq},
        \item distributed \gls{wfq}.
    \end{enumerate}
    \item distributed class-based \gls{wfq}
    \item \gls{ip} \gls{rtp} priority
    \item \gls{llq}
\end{enumerate}
