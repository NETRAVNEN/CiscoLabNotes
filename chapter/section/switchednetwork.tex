\section{Switch Network}

\subsection{VTP}
\fig{vtp/implementing-vtp}{imp-vtp1}{VTP}

\subsubsection{VTP Modes}
The tree modes a \gls{vtp} \textit{enabled} device can operate are
\begin{itemize}
    \item Transparent
    \item Server
    \item Client
\end{itemize}
Of course you can \textit{disable} \gls{vtp} altogether.

Key things to be aware of \textit{before} enabling \gls{vtp} in your environment is to make double sure of only having 1 \gls{vtp} domain. \textbf{If} 2 or more \gls{vtp} domains exists. Be triple sure to separate them! As to avoid having an \gls{vtp} server \gls{db} overridden with data from another \gls{vtp} domain.

The three \gls{vtp} modes \textit{operates} as follow
\begin{itemize}
    \item Transparent
    \begin{itemize}
        \item Creates, modifies and deletes \textit{local} \gls{vlan} only
        \item Forwards advertisements
        \item Does \textit{not} synchronizes vlan configurations.
    \end{itemize}
    \item Server
    \begin{itemize}
        \item Creates, modifies and deletes vlans
        \item Sends and forwards advertisements
        \item Synchronizes vlan configurations
    \end{itemize}
    \item Client
    \begin{itemize}
        \item Cannot create, modify or delete vlans
        \item Send and forwards advertisements
        \item Synchronizes vlan configurations
    \end{itemize}
\end{itemize}

\subsubsection{VTP Announcement}
\gls{vtp} operates with announcements sent out in intervals. Summarized it amounts to
\begin{itemize}
    \item 1 \textit{summary} announcement per 5th minute from the server
    \item The summary announcement informs clients of the current revision
    \item An announcement is sent out \textit{on the spot} when a change has been made on the\gls{vtp}server
\end{itemize}

Do remember it is \textbf{only} the \gls{vtp} server which has the \gls{vlan} configuration stored \textbf{on disk}. All device clients and transparent nodes do only store the \gls{vlan} delegated by \gls{vtp} in memory.

\subsubsection{Common Issues}
\begin{itemize}
    \item Different/Incompatible \gls{vtp} versions
    \item Wrong password
    \item Incorrect mode name
    \item No server set (all devices configured in transparent/client/\gls{vtp} disabled mode)
\end{itemize}

\subsubsection{VTP Versions}
\begin{itemize}
    \item Version 1
    \item Version 2
    \begin{itemize}
        \item Version-dependent	transparent	mode
        \item Consistencycheck
        \item Token ring support
        \item Unrecognized type-length-value support
    \end{itemize}
    \item Version 3 (not "yet" common)
    \begin{itemize}
        \item Extended \gls{vlan} support: Allow ranges are 1-1005,1018-2095. Not mentioned \gls{vlan} ranges up to 4095 is still reserved.
        \item Domain name is not automatically learned.
        \item Better security.
        \item Better database propagation.
        \item \gls{mst} now supported.
    \end{itemize}
\end{itemize}

\subsubsection{VTP Pruning}
The art of only allowing the \gls{vlan} traffic to flow on \textit{necessary} links.

This means if there are no clients in a \gls{vlan} on a device. Then no traffic for the inactive \gls{vlan}s are send down-/upstream on the link in question.

\fig{vtp/vtp-pruning}{vtpruning1}{VTP Pruning}

\subsubsection{Security}
It is \textbf{strongly} recommended to enable the security features supported in \gls{vtp}.

\textbf{Password:} MD5 hashing, Case-sensitive, Length between 8 and 64 chars.

\notice{VTP Scaling}{
As the network grows and grows and grows and grows some more over long/short timespans.
You will \textbf{for certain} come to cross-rode, where you \textbf{must} consider to
go away from using \gls{vtp} in the network. The problems of managing an elderly network and
wiping and re-introducing nodes in the network. You \textbf{will} face the issue of a
wiped vlan database from the \gls{vtp} domain.
}

\subsubsection{Example configuration}
\lstinputlisting{code/vtp/example.cfg}

\subsection{Channel Bundling (aka. EtherChannel, PortChannel)}
Channel bundling is the "art" of using multiple physical links as one single logical link in when viewed from the perspective of the forwarding plane.

Technologies:
\begin{itemize}
    \item \textbf{\gls{pagp}:} The Cisco-only thingy
    \item \textbf{\gls{lacp}:} The \gls{ieee} standard
    \item \textbf{Static:} Just forced on
\end{itemize}

\fig{channelbundling/network-without-channelbundling}{noethernetchannel}%
{No Channelbundling present}

Channel bundling of switch ports in the network may or may not be the best idea, in regards to the networks growth rate in terms of min. required bandwidth.

Channel bundling spreads out the in and egress flows based upon one of several methods configured on the switch:
\begin{itemize}
    \item Source to Destination \gls{mac}
    \item Source to Destination \gls{ip}
\end{itemize}
Keep in mind this will by no means archive true load balancing. Where all links are equally used based upon number of flows \textit{or} in terms of used bandwidth.

\begin{table}[h]
    \centering
    \caption{Channel bundling mechanisms}
    \label{chbundmech1}
    \resizebox{\columnwidth}{!}{%
        \begin{tabular}{|l|l|l|}
            \hline
            Hash Input Code & Hash Input Detecision & Switch Model \\ \hline
            dst-ip & Dest \gls{ip} addr & All models \\ \hline
            dst-mac  & Dest \gls{mac} addr & All models \\ \hline
            src-dst-ip & Src and dest \gls{ip} addr & All models \\ \hline
            src-dst-mac & Src and dest \gls{mac} addr & All models \\ \hline
            src-ip & Src \gls{ip} addr & All models \\ \hline
            src-mac & Src \gls{mac} addr & All models \\ \hline
            src-port & Src port no & 4500,6500 \\ \hline
            dst-port & Dest port no & 4500,6500 \\ \hline
            src-dst-port & Src and dest port no & 4500,6500 \\ \hline
        \end{tabular}%
    }
\end{table}

\fig{channelbundling/network-with-channelbundling}{withethernetchannel}%
{Channelbundling present}

\subsubsection{Protocol Properties}

\begin{itemize}
    \item \gls{lacp}
    \begin{itemize}
        \item Active: Enabled
        \item Passive: Waits for \gls{lacp} packets on the wire before enabled
    \end{itemize}
    \item \gls{pagp}
    \begin{itemize}
        \item Desirable: Enabled
        \item Auto: Waits for \gls{pagp} packets on the wire before enabled
    \end{itemize}
\end{itemize}

Some other \underline{required} settings to be (equal across all ports) aware of when configuring Channel bundling are
\begin{enumerate}
    \item Port speeds
    \item Duplex mode
    \item Configured \gls{vlan} ranges
\end{enumerate}

\subsubsection{Example configuration}
\lstinputlisting{code/channelbundling/example.cfg}