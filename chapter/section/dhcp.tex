\section{DHCP Process}

\fig{dhcp/dhcpdiscoverprocess}{dhcpdiscoverprocess}{DHCP Discover Process}

\subsection{DHCP Messages}

\begin{itemize}
    \item \textbf{DHCPDECLINE:} Message sent from the client to the server that the address is already in use.
    \item \textbf{DHCPNAK:} The server sends a refusal to the client for request for configuration.
    \item \textbf{DHCPRELEASE:} Client tells a server that it is giving up a lease.
    \item \textbf{DHCPINFORM:} A client already has an IP address but is requesting other configuration parameters that the DHCP server is configured to deliver such as DNS address.
\end{itemize}

\section{DHCP Options}

\begin{itemize}
    \item \textbf{43} Vendor-encapsulated option that enables vendors to have their own list of options on the server.
    \item \textbf{69} SMTP server, if you want to specify available SMTP servers to the client.
    \item \textbf{70} POP3 server, if you want to specify available POP3 servers to the client.
    \item \textbf{150} TFTP server that enables your phones to access a list of TFTP servers.
\end{itemize}

\section{DHCP Example Configuration}

\subsection{Cisco}

\begin{txt}
ip dhcp excluded-address 192.168.0.254
!
ip dhcp pool LAN-1-POOL-DHCP
 network 192.168.0.0 255.255.255.0
 default-router 192.168.0.254
 lease 2 ! set in days
\end{txt}

When configuring a Layer 3 interface as a relay port for DHCP request for a subnet. Set the ip helper command on the interface with one \textit{or} more ip addresses.

\begin{txt}
interface GigabitEthernet 0/3
 ip helper-address 192.168.220.220
 ip helper-address 192.168.222.222
\end{txt}
