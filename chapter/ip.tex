\chapter[IP]{Internet Protocol}

\section[IPv6]{Internet Protocol v6}

\gls{ipv6} has recently been defined in an updated \rfc{8200} which obsoletes several of the older \gls{rfc} documents regarding \gls{ipv6}.

\gls{ipv6} came about in 1998 when the original \rfc{2460} was published. It aims to provide globally route-able addresses (i.e. no need for \gls{nat}) and provides a hierarchical way to allocate address prefixes in a way which makes it simple to do route aggregation\footnote{This helps limits the size of the Internet's global routing table!}.

\fig{ipv6_address_leading_zeros.svg}{ipv6zeroes}{An illustration of an example IPv6 address with leading zeros in the binary rendering}



\section[IPv4]{Internet Protocol v4}