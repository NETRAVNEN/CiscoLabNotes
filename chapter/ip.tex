\chapter[IP]{Internet Protocol}

\section[IPv6]{Internet Protocol v6}

\gls{ipv6} has recently been defined in an updated \rfc{8200} which obsoletes several of the older \gls{rfc} documents regarding \gls{ipv6}.

\gls{ipv6} came about in 1998 when the original \rfc{2460} was published. It aims to provide globally route-able addresses (i.e. no need for \gls{nat}) and provides a hierarchical way to allocate address prefixes in a way which makes it simple to do route aggregation\footnote{This helps limits the size of the Internet's global routing table!}.

\wikicommons[An illustration of an example IPv6 address with leading zeros in the binary rendering]{ipv6_address_leading_zeros}

\subsection{IP to client}

Several ways to assign a client an address exists.
\begin{itemize}
    \item Let the client handle it itself (i.e. \gls{dhcp} server present): \gls{slaac}.
    \begin{itemize}
        \item \gls{icmp6} router discovery messages is used to detect info 'bout the connected network segment.
        \item Upon \textit{link up} client sends link-local solicitation multicast req. for network parameters,
        \item router \textit{responds with}  router adv. packet cont. \gls{ip} cfg parameters.
    \end{itemize}
    \item Let the \gls{dhcp} server assign \textit{partial} info to the client.
    \item Let the \gls{dhcp} server assign \textit{every} info to the client.
\end{itemize}

\textbf{Privacy} is a large concern regarding \gls{ipv6} because of the globally unique address the client posses.

Implementation to do privacy regarding the host bits of an \gls{ip6} has been done to protect the clients (and users) from being tracked. Alas, if the \gls{isp} do static prefix assignments to end users. This privacy protection may be somewhat unusable. As the network prefix will always remain the same. Regardless of the host-bits being changed often.

Have 3 different forms:
\begin{enumerate}
    \item 2001:0db8:0000:0000:0000:ff00:0042:8329,
    \item 2001:db8:0:0:0:ff00:42:8329, {\footnotesize (i.e. remove leading zeroes per group delimited by colon)}
    \item 2001:db8::ff00:42:8329. {\footnotesize (i.e. remove groups containing all zeroes in succession after each other) (only done \textit{once!}}
\end{enumerate}

\subsection{Packet Header}

\wikicommons{Ipv6_header}

\subsection{Address Types}

\begin{itemize}
    \item \itemhead[]{Link-Local}
    \begin{itemize}
        \item Address assigned from the fe80::/7 prefix.
        \item Either derived with the EUI-64\footnote{The EUI-64 involves the MAC address and injecting fffe into the middle making it 64 bits and using this as host bits} method or randomly selected. Then assigned after \gls{dad} has been run on the network segment.
    \end{itemize}
    \item \itemhead[]{Global Addressing}
\end{itemize}

\subsection{Multicast}

\section[IPv4]{Internet Protocol v4}