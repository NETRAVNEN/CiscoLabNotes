\chapter{NTP}

\gls{ntp} is the source of all evil and \gls{sla}. A network wide source of time configuration for all network nodes, servers, clients etc. is necessary.

\textbf{Configure timezone}\\In this case it\tsq{s} for \gls{metdst}\textbf{:}

\begin{txt}
clock timezone MET 1 0
clock summer-time MET-DST recurring last Sun Mar 2:00 last Sun Oct 3:00
\end{txt}

\textbf{Configure used timezone}\\when doing logging and debugging operations\textbf{:}

\begin{txt}
service timestamps debug datetime msec localtime show-timezone
service timestamps log datetime msec localtime show-timezone
\end{txt}

A select number of Cisco switches support synchronization with the hardware clock, too. The standard is to only sync the software clock.\\\cliline{ntp update-calendar}

\fig{ntp/ntp}{ntp}{\gls{ntp}}

\gls{ntp} servers are a hierarchical tree with stratum 0 servers as the authoritative in the tree. These servers get their time from either \gls{gprs} satellites or atomic clocks {\footnotesize (i.e. an authoritative time \gls{src})}.

\subsection{Characteristics}

\begin{itemize}
	\item Uses \gls{udp} port 123 on both \gls{src} and \gls{dst},
	\item polling interval ranging from 64-1024 sec. Length of interval is dependant upon network cond.,
	\item large differences between \gls{ntp} reference time and local client time will result in increased pooling interval.
\end{itemize}

\fig{ntp/ntpstratum}{ntpstratum}{Stratum levels}



\section{The old NTP from \tsq{85}}

\section{Secure NTP}

\subsection{Characteristics}

\begin{itemize}
	\item 
\end{itemize}
