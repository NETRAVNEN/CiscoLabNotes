% !TeX TS-program =
% !TeX spellcheck = en_DK
% !TeX encoding = UTF-8
% !TeX root = ../main.tex

\chapter{L2 to L3}

\section{Vlan-to-vlan routing}

\myquote{}{Guidance and Understanding of the art of Layer 3 networks. Routing between different slash 24\tsq{s}.\\ \textit{Aka. Inter-vlan routing.}}

There are different ways to go \tsq{bout} Inter-vlan routing and doing it.

\begin{itemize}
    \item \textbf{Some are using:}
    \begin{enumerate}
        \item external router,
        \item switch virtual interfaces\footnote{\texttt{Switches interface -> Vlan interfaces}},
        \item routed ports\footnote{\texttt{Routed interface -> Subinterfaces}},
        \item bridge virtual interface\footnote{\texttt{L3 bridging interface}}
    \end{enumerate}
    \item \textbf{Ways to do it:}
    \begin{enumerate}
        \item router-on-a-stick if the network is running collapsed core or not have a distribution switch in the middle.
        \item (...)
    \end{enumerate}
\end{itemize}

\pagebreak

\subsection{Interface configs}

\subsubsection{Routed interfaces}

\begin{txt}
    interface GigabitEthernet 0/1.10
    encapsulation dot1q 10
    ip address 192.168.0.1 255.255.255.128
    !
    interface GigabitEthernet 0/1.20
    encapsulation dot1q 20
    ip address 192.168.0.129 255.255.255.128
\end{txt}

\subsubsection{Switches interfaces}

\begin{txt}
    Vlan10
    name VLAN10
    Vlan20
    name VLAN20
    !
    interface Vlan10
    ip address 192.168.1.1 255.255.255.128
    interface Vlan20
    ip address 192.168.1.129 255.255.255.128
    !
    interface GigabitEthernet 0/2
    switchport mode trunk
    switchport trunk encapsulation dot1q
    switchport trunk allowed vlan 10,20
\end{txt}

\pagebreak

\fig{network/routeronastick}{routeronastick}{\bsq{Router on a Stick} concept.}

\pagebreak

\subsection{? Troubleshooting ?}

\begin{enumerate}
    \item \textbf{Missing VLAN:}
    \begin{itemize}
        \item VLAN might not be defined across all the switches.
        \item VLAN might not be enabled on the trunk ports.
        \item Ports might not be in the right VLANs.
    \end{itemize}
    \item \textbf{Layer 3 interface misconfiguration:}
    \begin{itemize}
        \item Virtual interface might have the wrong IP address or subnet mask.
        \item Virtual interface might not be up.
        \item Virtual interface number might not match with the VLAN number.
        \item Routing has to be enabled to route frames between VLAN.
        \item Routing might not be enabled.
    \end{itemize}
    \item \textbf{Routing protocol misconfiguration:}
    \begin{itemize}
        \item Eyery interface or network needs to be added in the routing protocol.
        \item The new interface might not be added to the routing protocol.
        \item Routing protocol configuration is needed only if VLAN subnets need to communicate to the other routers, as previously
        mentioned in this chapter.
    \end{itemize}
    \item \textbf{Host misconfiguration:}
    \begin{itemize}
        \item Host might not have the right IP address or subnet mask.
        \item Each host has to have the default gateway that is the SVI or Layer 3 interface to communicate with other networks and VLAN.
        \item Host might not be configured with the default gateway.
    \end{itemize}
\end{enumerate}
